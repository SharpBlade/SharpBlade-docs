\chapter{App flow in SharpBlade}
If you've read the documentation for Razer's SDK, you will have noticed the application flow that is listed there: \c|RzSBStart() -> RzSBQueryCapabilities() -> Register gestures/keys -> <app stuff> -> RzSBStop()|.

SharpBlade has a similar flow when creating applications, but in a way that should be familiar to C\# developers.

\section[The RazerManager class]{The \c|RazerManager| class}
\c|RazerManager| is the core of SharpBlade, it's what manages and provides all the other components like the \c|Touchpad| class and various \c|DynamicKey| instances. The class itself is a singleton, so you obtain a reference to it via the \c|Instance| property.

Like other singletons, this is done to enforce a single instance of \c|RazerManager| is used for an application. For your convenience, you may want to save a local reference to \c|RazerManager| like so:

\begin{csexample}{getting-instance}{Getting an instance of RazerManager}
    var manager = SharpBlade.RazerManager.Instance;
\end{csexample}

This (assuming it's the first access to \c|Instance|) will initialise the internal field with a new instance of \c|RazerManager|, which calls the necessary native functions to initialise the SDK. You can then use this manager instance to access the various parts of the SDK and perform method calls.

\section{Life cycle}
As mentioned, \c|RzSBStart| is called internally when the \c|Instance| property is first accessed, but when is \c|RzSBStop| called then? On dispose!

\c|RazerManager| implements the \c|IDisposable| interface, and calls \c|RzSBStop| when it disposes, either explicitly by code or when it's collected by the GC.
