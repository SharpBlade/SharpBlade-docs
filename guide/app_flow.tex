\chapter{App flow in SharpBlade}
If you've read the documentation for Razer's \gls{sdk}, you will have noticed the application flow that is listed there: \c|RzSBStart() -> RzSBQueryCapabilities() -> Register gestures/keys -> <app stuff> -> RzSBStop()|.

SharpBlade has a similar flow when creating applications, but in a way that should be familiar to C\# developers.

\section[The RazerManager class]{The \sbapi[T_SharpBlade_Razer_RazerManager]{RazerManager} class}
\sbapi[T_SharpBlade_Razer_RazerManager]{RazerManager} is the core of SharpBlade, it's what manages and provides all the other components like the \sbapi[T_SharpBlade_Razer_Touchpad]{Touchpad} class and various \sbapi[T_SharpBlade_Razzer_DynamicKey]{DynamicKey} instances. The class itself is a singleton, so you obtain a reference to it via the \sbapi[P_SharpBlade_Razer_RazerManager_Instance]{RazerManager.Instance} property.

Like other singletons, this is done to enforce a single instance of \sbapi[T_SharpBlade_Razer_RazerManager]{RazerManager} is used for an application. For your convenience, you may want to save a local reference to \sbapi[T_SharpBlade_Razer_RazerManager]{RazerManager} like so:

\begin{csexample}{getting-instance}{Getting an instance of RazerManager}
    var manager = SharpBlade.RazerManager.Instance;
\end{csexample}

This (assuming it's the first access to \sbapi[P_SharpBlade_Razer_RazerManager_Instance]{RazerManager.Instance}) will initialise the internal field with a new instance of \sbapi[T_SharpBlade_Razer_RazerManager]{RazerManager}, which calls the necessary native functions to initialise the \gls{sdk}. You can then use this manager instance to access the various parts of the \gls{sdk} and perform method calls.

\section{Life cycle}
As mentioned, \rzapi{RzSBStart} is called internally when the \sbapi[P_SharpBlade_Razer_RazzerManager_Instance]{RazerManager.Instance} property is first accessed, but when is \rzapi{RzSBStop} called then? On dispose!

\sbapi{RazerManager} implements the \c|IDisposable| interface, and calls \rzapi{RzSBStop} when it disposes, either explicitly by code or when it's collected by the \gls{gc}.
